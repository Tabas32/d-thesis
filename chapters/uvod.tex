% !TEX root = ../thesis.tex

\chaptermark{Úvod}
\addcontentsline{toc}{chapter}{Úvod}

\chapter*{Úvod}

Často sa hovorí o konci platnosti Moorovho zákona. Je možné, že v blízkej
budúcnosti svet bude nútený zmentiť klasické počítače od základu. Jedným 
často spomínaným vývojovým schodom v tejto oblasti je kvantový počítač.
Pokusy využiť kvantovú fyziku v odbore počítačovej vedy možno nájsť už v
minulosti, no až v horizonte niekoľkých rokov nastal prelom a kvantové
počítače vznikajú po celom svete.

No napriek tomu stále chýba množstvo nástrojv, ktoré by sprístupnili vývoj
širšej verejnosti. Existujú voľne dostupné simulátory na generovanie kvantových
obvodov, ale od plne funkčných programov máme ešte ďaleko. Tento odbor je 
veľmi náročný a každý pokus zaberá množstvo času. Aj ten najjednoduchší program
je nutné zložito vytvárať pomocou internetových nástrojov, nehovoriac o 
prístupe k reálnemu stroju.

Touto prácou sa pokúsime vylepšiť súčasnú situáciu. A to vytvorením nástroja,
ktorý by zlepšil porozumenie pri vykonávaní programov. I keď nepôjde o 
dokonale sofistikovaný simulátor kvantového počítača, napriek tomu porozumenie,
zrýchlenie a spríjemnenie vývoja programov pre tento druh počítačov môže 
priniesť nové pokroky v odvetví. Našim cieľom je navrhnúť simulátor tak, aby 
bolo prirodzene jednoduché zistiť v akom stave sa kvantový systém nachádza.

Bude v našom úmysle, čo najjednoduchšie vysvetliť princípy, ktoré sa skrývajú
za fungovaním kvantových počítačov. Naša práca ponúka teoretické minimum
nutné na porozumenie praktických experimentov a využíva ho pri priblížení
základných kvantových javoch, ako napríklad zvjazanie kvantových bitov, ktoré 
tvoria podstatu kvantových výpočtov ale aj pri zložitejších úkonoch ako 
kvantová teleportácia.


Čo je najdôležitejšie, pokúsime sa jasne a zrozumiteľne vysvetliť princíp
merania zmien stavov kvantových bitov. Poskytneme pohľad do matematického
aparátu, ktorý umožňuje simuláciu kvantových programov na klasických strojoch.
Priblížime vývoj pravdepodobnostného modelu vytvoreného vo funkcionálnom jazkyu
Haskell. A poskytneme komplexný návod na jeho použitie.
