% !TEX root = ../thesis.tex

\chapter{Systémová príručka}
%QCircuits.hs  
%QDefinitions.hs
%QEntanglement.hs  
%QMatrix.hs  
%QOperations.hs  
%QWriter.hs

Všetky potrebné skripty sa nachádzajú v adresáry program. Pri načítaní modulu
QWriter sa automaticky načítajú aj ostatné. 
\begin{itemize}
    \item QCircuits.hs - obsahuje pravdepodobnostý model a celú logiku merania,
    \item QDefinitions.hs - obsahuje definície základných kvantových stavov a
hradiel,
    \item QEntanglement.hs - obsahuje funkcie pre výpočet alfa a beta normy,
    \item QMatrix.hs - obsahuje oprerácie pre prácu s maticami,
    \item QOperations.hs - obsahuje definície operácií s kvanotvými bitmi
    \item QWriter.hs - obsahuje funkciu main a proces prekladu a zápisu do
latexu.
\end{itemize}

\section{Popis algoritmu pravdepodobnostného modelu}

Algoritmus je založený na rekurzívnej aktivácií funkcie modelu nad obvodom.
Každý level je postupne spracovaný pričom sa vytovrý nový strom stavov, hradlá
daného levelu sú aplikované na listy priebežného stromu stavov kvantových bitov
a tabuľka výsledkov je aktualizovaná, ak je level označený pre meranie.

Tvorba nového stromu je závislá na hradlách, ktoré level obsahuje. Ak v leveli
nastáva previazanie niektorých z bitov pomocou hradla CNOT, je nutné rozvetviť
strom na dva podstromi. Tieto vetvy určujú možnosti, kde kontrólne bity sú
v stavoch \(\ket{1}\) a teda nastane preklopenie cieľového bitu a možnosť
kde preklopenie nenastane.

\section{Popis implementácie}

Ukladanie stavov kvantových bitov zabezpečujú dátové štruktúry
\begin{code}
type QBit = [[Complex Double]]
type States = [QBit]
type SubTrees = [StateTree]

data StateTree = StateTree Double States SubTrees deriving Show
\end{code}

Kvantový bit nie je nič iné ako matica komplexných čísel a typ States je potom
list týchto bitov. Strom stavov obsahuje 3 časti. Pravdepodobnosť vykonania,
typu Double, list stavov koreňa tohto stromu a odkaz na potomkov. List stromu
je teda takisto strom ale nemá žiadnych potomkov.

Tabuľka výsledkov je dvojrozmerným poľom, ktoré obsahuje v riadkoch merania 
v jednotlivých časových okamihoch. Stĺpce tejto tabuľky obsahujú štruktúru, 
ktorá obaľuje pravdepodobnosť dosiahnutia daného stavu systému a list bitov 
do ktorých systém kolabuje.
\begin{code}
data R = R Double [Int]
    deriving (Show)
type T = [R]
type Ts = [T]
\end{code}

Pre jednoduchšiu prácu pravdepodobnostný model pracuje so štruktúrov, ktorá 
zaobaľuje tieto entity.
\begin{code}
data RT = RT StateTree Ts
    deriving (Show)
\end{code}

Posledným datovým typom, ktorý je nutné spomenúť je reprezenácia kvantového
obvodu.
\begin{code}
data Element = X
    | Y
    | Z
    | H
    | S
    | Sd
    | T
    | Td
    | Ct   -- CNOT target bit
    | Cc  -- CNOT control bit
    | E   -- empty
    deriving (Show, Eq)

type LevelGates = [Element]
data Level = Level LevelGates Bool deriving Show
type Circuit = [Level]
\end{code}

Obvod je reprezentovaný ako list vertikálnych levelov. Každý level je zostavený
z hradiel (typu LevelGates) a indikátora, ktorý označuje či má biť daný level
meraný (typu Bool). Jednotlivé hradlá levelu sú typu Element a obsahujú
základné kvantové hradlá, pričom Ct označuje cieľový bit hradla CNOT, Cc 
označuje kontrólny bit hradla CNOT a E označuje absenciu hradla na danej
pozícii. Ostané hradlá je prirodzene jasné odlíšiť.

Nasleduje popis jednotlivých funkcií.

\subsection*{processCircuit :: Circuit -> RT -> RT }
\textbf{Vstup:} Na vstupe je kvantový obvod a tabuľka výsledkov.\\
\textbf{Výstup:} Nová tabuľka výsledkov s pravdepodobnosťami meraní a stromom
stavov kvantových bitov. \\
\textbf{Popis:}
Hlavná funkcia merania obvodu. Rekurzívne prechádza list levelov. Ak je 
level meraný, aktualizuje tabuľku Ts. Takisto vyvolá aktiváciu hradiel z 
levela na strome stavov.

\subsection*{upTs :: Ts -> StateTree -> Ts}
\textbf{Vstup:} Tabuľka výsledkov a strom stavov.\\
\textbf{Výstup:} Nová tabuľka so skolabovanými stavmi.\\
\textbf{Popis:} Funkcia vyvolá kolabovanie listov stromu stavov a zapíše 
výsledok do tabuľky výsledkov.

\subsection*{collapseR :: StateTree -> T}
\textbf{Vstup:} Strom stavov.\\
\textbf{Výstup:} Namerané výsledky z jedného časového okamihu.\\
\textbf{Popis:} Funkcia skolabuje všetky kvantové bity v listoch stromu 
stavov a vráti list týchto meraní.

\subsection*{isLMeasured :: Level -> Bool}
\textbf{Vstup:} Level.\\
\textbf{Výstup:} Inikátor merania.\\
\textbf{Popis:} Funkcia vráti True ak má byť level meraný, False inak.

\subsection*{processLevel :: Level -> StateTree -> StateTree}
\textbf{Vstup:} Level, ktorý sa má spracovať a strom stavov.\\
\textbf{Výstup:} Nový strom stavov.\\
\textbf{Popis:} Funkcia rekurzívne precháza stromom stavov až kím nedosiahne
koncové listy. Pri listoch vyvolá funkciu na aktivovanie hradiel levelu na 
daných bitoch. Nové stavy sú zapísané do nového stromu stavov kvantových bitov.

\subsection*{calculateStateTrees ::  LevelGates -> States -> Double -> [StateTree]}
\textbf{Vstup:} Hradlá levelu, ktorý sa aplikuje, Stavy bitov, ktoré hradlami
prechádzajú a pravdepodobnosť nastatia tejto situácie.\\
\textbf{Výstup:} Nové stromy stavov, ktoré budú pridané ako listy do hlavého
stromu.\\
\textbf{Popis:} Ak list hradiel obsahuje hradlo CNOT funkcia vyvolá tvorbu
dvoch stromov pre 2 prípady, ktoré môžu nastať. V inom prípad funkcia zavolá
aplikáciu hradiel na jednotlivé kvantové bity a nové stavy vráti v novom 
podstome.

\subsection*{applyCNot ::  LevelGates -> States -> Double -> [StateTree]}
\textbf{Vstup:} Hradlá levelu, ktorý sa aplikuje, Stavy bitov, ktoré hradlami
prechádzajú a pravdepodobnosť nastatia tejto situácie.\\
\textbf{Výstup:} List s dvoma stromami stavov.\\
\textbf{Popis:} Funkcia prepočíta pravdepodobnosť s akou kontrólne bity 
kolabujú do stavu \(\ket{1}\). Ak je táto pravdepodobnosť nulová, je 
vytvorený jediný podstrom a nie sú aplikované žiadne hradlá. V opačnom prípade
sú vytvorené stromy dva, kde jeden aplikuje preklopenie cieľového bitu hradla
CNOT a druhý nie. Vstupná pravdepodobnosť je využitá pri prepočte 
pravdepodobnosti nastatia nových podstromov.

\subsection*{cnPasP :: [Element] -> [QBit] -> Double}
\textbf{Vstup:} Hradlá levelu a stavy bitov na ktorých sa dané hradlá 
aplikujú.\\
\textbf{Výstup:} Pravdepodobnosť.\\
\textbf{Popis:} Funkcia prepočítava pravdepodobnosť s akou kontrólne bity 
hradla CNOT kolabujú do stavu \(\ket{1}\).

\subsection*{probCt1 :: Element -> QBit -> Double}
\textbf{Vstup:} Hradlo a bit, ktorý daným hradlom prechádza.\\
\textbf{Výstup:} Pravdepodobnosť bitu dosiahnutia stav \(\ket{1}\).\\
\textbf{Popis:} Ak je na vstupe hradlo Cc (kontrólny bit hradla CNOT) funkcia
vracia pravdepodobnosť daného bitu kolabovať do stavu \(\ket{1}\). Pre 
akékoľvek iné hradlo, funkcia fracia -1. Funkcia sa používa pri prepočťe 
cnPasP.

\subsection*{applyGate :: Element -> QBit -> QBit}
\textbf{Vstup:} Hradlo a bit, ktorý daným hradlom prechádza.\\
\textbf{Výstup:} Nový stav po prechode hradlom.\\
\textbf{Popis:} Funkcia vykonáva operáciu |* so správnymi hradlami. Slúži
vpodstate na preklad z typu Element na typ [[Complex Double]].

\subsection*{toBin :: Int -> [Int]}
\textbf{Vstup:} Celé číslo v desiatkovej sústave.\\
\textbf{Výstup:} List reprezentujúci binárne číslo.\\
\textbf{Popis:} Funkcia slúži na prevod medzi desiatkovov a dvojkovov 
sústavov.

\subsection*{formatBin len n}
\textbf{Vstup:} Požadovaná dĺžka. \\
\textbf{Výstup:} Binárne číslo.\\
\textbf{Popis:} Funkcia formátuje binárne číslo na správnu veľkosť. Pridáva
nuly na začiatok, až kým číslo nemá správnu veľkosť.

\subsection*{createResList :: Int -> [[Int]]}
\textbf{Vstup:} Počet kvantových bitov v obvode.\\
\textbf{Výstup:} Tabuľka možných výsledkov.\\
\textbf{Popis:} Funkcia slúži na vytovenie všetkych možností n bitového 
binárneho čísla. Vracia všetky možné prípady, ktoré by kolabovaním mohli
nastať.

\subsection*{collapseStates :: States -> T}
\textbf{Vstup:} Stavy, ktoré majú byť kolabované.\\
\textbf{Výstup:} Meranie v danom časovom úseku.\\
\textbf{Popis:} Funkcia vytvorí list s výsledkami pre každú možnosť, ktorá
môže nastať.

\subsection*{createRList :: States -> [Int] -> R}
\textbf{Vstup:} Stavy, ktoré majú byť kolabované a binárne číslo popisujúce
skolabovaný stav.\\
\textbf{Výstup:} Výsledok pre daný stav.\\
\textbf{Popis:} Funkcia slúžiaca na konštrukciu výsledku pre daný stav.

\subsection*{calculateProbs :: States -> [Int] -> [Double]}
\textbf{Vstup:} Stavy, ktoré majú byť kolabované a binárne čáslo popisujúce
skolabovaý stav.\\
\textbf{Výstup:} List pravdepodobností.\\
\textbf{Popis:} Funkcia zavolá prepočet pravdepodobnosti kolabovania do stavov
\(\ket{0}\) alebo \(\ket{1}\) a vráti ich list.

\subsection*{calcP :: QBit -> Int -> Double}
\textbf{Vstup:} Stav kvantového bitu a požadovaný skolabovaný stav.\\
\textbf{Výstup:} Pravdepodobnosť skolabovania do daného stavu.\\
\textbf{Popis:} Vracia pravdepodobnosť, že daný kvantový bit skolabuje do 
daného stavu 0 alebo 1.

\subsection*{isRgt0 :: R -> Bool}
\textbf{Vstup:} Výsledok.\\
\textbf{Výstup:} Inikátor možnosti dosiahnutia daného výsledku.\\
\textbf{Popis:} Funkcia vracia True ak pravdepodobnosť daného výsledku je 
viac ako 0.

\subsection*{mulR :: Double -> R -> R}
\textbf{Vstup:} Násobiteľ a výsledok, ktorý má byť násobený.\\
\textbf{Výstup:} Vynásobený výsledok.\\
\textbf{Popis:} Vracia nový výsledok, kde pôvodná pravdepodobnosť je vynásobená
vstupom.
