% !TEX root = ../thesis.tex

\chapter{Teoretické základy kvantových systémov}

V nasledujúcej kapitole sú uvedené poznatky z teórie kvantových výpočtov a kvantových obvodov.

\section{Základné definície}

\begin{itemize}
\item[] \textbf{Hilbertov priestor} \\
\label{hil_space}
Hilbertov priestor (angl. Hilber space) je úplný konečnorozmerný vektorový priestor, v ktorom je definovaná operácia skalárneho súčinu \(\Braket{u|v}\), kde \(u,v\) sú \(N\)-rozmerné vektory s komplexnými zložkami \cite{Nie10}.
Konečnorozmerným vektorovým priestorom nazývame taký priestor, ktorého báza je množina lineárne nezávislých  vektorov, a ktorá generuje celý tento priestor.
Pre úplný priestor platí, že existuje Cauchyho postupnosť, ktorou je dosiahnuteľný ľubovoľný stav, charakterizovateľný \(N\)-rozmerným vektorom \(\Ket{\psi} \in \field{C}^{N}\), ktorý je vždy normalizovaný.

\item[] \textbf{Unitárne zobrazenie} \\
Unitárne zobrazenie (angl. Unitary map) je rotáciou, čiže zmenou ortonormálnej bázy.

\item[] \textbf{Kvantový bit} \\
Za kvantový bit je možné považovať objekt, ktorý popisuje stav kvantového systému \cite{Schu95}.
Z matematického pohľadu je to vektor v dvojrozmernom Hilberovom priestore \(\field{C}^{2}\).
No v reále ide o fotón.
Budeme sa zaoberať dvojstavovými kvantovými systémami, kde je fotón nútený skolabovať do jedného z dvoch stavov.
A teda vektor, ktorý bude popisovať tento kvantový bit vyjadríme ako \(u = \binom{\alpha}{\beta}\), \( ( \alpha, \beta \in \field{C} ) \) a \(u \in \field{C}^{2}\) \cite{Kay07}.
No vhodnejším sa javí vyjadrenie tohto vektora superpozíciou, teda lineárnou kombináciou základných stavov \(\Ket{0}, \Ket{1}\), ktoré zodpovedajú klasickým bitom \(0,1\).
Teda \[u = \binom{\alpha}{\beta} = \alpha \binom{1}{0} + \beta \binom{0}{1} = \alpha\Ket{0} + \beta\Ket{1},\]
kde množina \(\{\Ket{0}, \Ket{1}\} = \{\binom{1}{0}, \binom{0}{1}\}\) je nazývaná základná báza.
Väčšinou je využívaná základná báza \(\{\Ket{0}, \Ket{1}\}\), no je možné sa stretnúť aj s bázami \(\{\Ket{+}, \Ket{-}\}\) a \(\{\Ket{\circlearrowright}, \Ket{\circlearrowleft}\}\). Tieto bázy sú dosiahnuteľné zo základnej bázy unitárnymi transformáciami.

\item[] \textbf{Superpozícia} \\
Superpozíciou (angl. superposition) dvoch vektorov je vyjadrený stav kvantového bitu \(\Ket{\psi}\), \(\Ket{\psi} \in \field{C}^2\).
Ide o lineárnu kombináciu, a teda vo všeobecnosti tieto vektory môžu byť dva ľubovoľné, no lineárne nezávislé vektory \(u\) a \(v\). Čiže
\[\Ket{\psi} = \alpha u + \beta v.\]
Pre kvantové výpočty, ale má väčší význam využitie ortonormálnych vektorov.
\[\Ket{\psi} = \alpha \Ket{0} + \beta \Ket{1},\]
\[\Ket{\psi} = \alpha \Ket{+} + \beta \Ket{-},\]
\[\Ket{\psi} = \alpha \Ket{\circlearrowright} + \beta \Ket{\circlearrowleft},\]
kde \(\alpha, \beta \in \field{C}\) a platí \(|\alpha|^2 + |\beta|^2 = 1\).

\item[] \textbf{Previazanosť kvantových bitov} \\
Majme stav dvoch qbitov
\[\Ket{\psi} = \frac{\Ket{00} + \Ket{11}}{\sqrt{2}}\]
Pre tento stav neexistuje taká dvojica stavov \(\Ket{a}\) a \(\Ket{b}\), že platí \(\Ket{\psi} = \Ket{a}\Ket{b}\).
Hovoríme, že stav zloženého systému, ktorý nemožno zapísať ako súčin stavov jeho komponentov sa nazýva previazaným (angl. entagled) stavom \cite{Nie+00}.

V prípade jednoduchého n-bitového kvantového systému môžme jeho celkový stav \(\Ket{\psi}\) vyjadriť tenzorovým súčinom vektorov stavov jednotlivých bitov \(\Ket{\psi_0}\), \(\Ket{\psi_1}\), \(\dots\), \(\Ket{\psi_{n-1}}\).
Čiže
\[\Ket{\psi} = \Ket{\psi_0} \otimes \Ket{\psi_1} \otimes \dots \otimes \Ket{\psi_{n-1}}\]
Toto však neplatí, ak dva alebo viac kvantových bitov je navzájom previazaných.
Pretože previazané bity sú charakteristické rovnakými vektormi, a to počas celého výpočtu a aj pri meraní.
\end{itemize}

\section{Systém s jedným kvantovým bitom}
Majme kvantový systém, ktorý obsahuje jediný kvantový bit. Označme ho \(\psi\)
a platí
\[\Ket{\psi} = \binom{\alpha}{\beta} = \binom{\alpha}{0} + \binom{0}{\beta} = \alpha \binom{1}{0} + \beta \binom{0}{1} = \alpha \Ket{0} + \beta \Ket{1}, \]
kde \(\alpha, \beta \in \field{C}\) a \(\Ket{\psi} \in \field{C}^{2}\).
Čiže stav kvantového systému \(\Ket{\psi}\) je superpozíciou stavov \(\Ket{0}\) a \(\Ket{1}\).

Používame Bra-ket zápis, ktorý bol vysvetlený v časti Vektory \ref{vektory}. 
Čiže platí to isté ako pre vektory
\[\Ket{\psi} = \binom{\alpha}{\beta}\]
\[\Bra{\psi} = (\alpha^{\dag}\beta^{\dag})\]
a teda normu tohto kvantového bit možno odvodiť zo
\[\Braket{\psi|\psi} = (\alpha^{\dag}\beta^{\dag})\binom{\alpha}{\beta} = \alpha^{\dag}\alpha + \beta^{\dag}\beta = |\alpha|^{2} + |\beta|^{2} = ||\Ket{\psi}||^{2}\]

a samozrejme platí
\[|\alpha|^{2} + |\beta|^{2} < 1\]

\section{Systém s viacerými kvantovými bitmi}
Majme kvantový systém, ktorý je zložený z troch nepreviazaných kvantových bitov.
Označme ich \(\psi_1\), \(\psi_2\) a \(\psi_3\) . Platí
\[\psi_1 = \alpha_1 \Ket{0} + \beta_1 \Ket{1}\]
\[\psi_2 = \alpha_2 \Ket{0} + \beta_2 \Ket{1}\]
\[\psi_3 = \alpha_3 \Ket{0} + \beta_3 \Ket{1}\]

Stav tohto systém možno odvodiť ako
\[\psi = \Ket{\psi_1 \psi_2 \psi_3} = \binom{\alpha_1}{\beta_1} \otimes \binom{\alpha_2}{\beta_2} \otimes \binom{\alpha_3}{\beta_3}\]
, a to sa rovná
\[
\begin{pmatrix}
    \alpha_1 \alpha_2 \alpha_3 \\
    \alpha_1 \alpha_2 \beta_3\\
    \alpha_1 \beta_2 \alpha_3\\
    \alpha_1 \beta_2 \beta_3\\
    \beta_1 \alpha_2 \alpha_3 \\
    \beta_1 \alpha_2 \beta_3\\
    \beta_1 \beta_2 \alpha_3\\
    \beta_1 \beta_2 \beta_3\\

 \end{pmatrix}
\]

\section{Princíp merania}
Pre príklad nám poslúži jednoduchý obvod s tromi nepreviazanými kvantovými 
bitmi \(\psi_1, \psi_2\) a \(\psi_3\). Každý s týchto bitov môže kolabovať do
jedného z dvoch stavov. A to \(\Ket{0}\) a \(\Ket{1}\) (\(\binom{1}{0}\) a
\(\binom{1}{0}\)). Platí
\[|\alpha_1\alpha_2\alpha_3|^2 + |\alpha_1\alpha_2\beta_3|^2 + \dots + |\beta_1\beta_2\beta_3|^2 = 1\]
Povedzme, že bit \(\psi_1\) skolabuje. Pravdepodobnosť s akou skolabuje do
stavu \(\binom{1}{0}\) vypočítame ako \(|\alpha_1|^2\). Pravdepodobnosť s akou 
skolabuje do stavu \(\binom{0}{1}\) vypočítame ako \(|\beta_1|^2\) \cite{Kit95}. Bez ohľadu
na to, aký stav tento kvantový bit nadobudne, pre kvantový systém bude platiť
\[|\alpha_2\alpha_3|^2 + |\alpha_2\beta_2|^2 + |\beta_2\alpha_3|^2 + |\beta_2\beta_3|^2 = 1\]

Ak bit \(\psi_1\) kolabuje do \(\Ket{0}\) tak sa systém bude nachádzať v tomto
stave
\[
\begin{pmatrix}
    \alpha_2 \alpha_3 &
    \alpha_2 \beta_3&
    \beta_2 \alpha_3&
    \beta_2 \beta_3&
    0&
    0&
    0&
    0\\
 \end{pmatrix}^T
\]
obdobne, ak nadobudne druhý stav
\[
\begin{pmatrix}
    0 & 0 & 0 & 0  & \alpha_2 \alpha_3 & \alpha_2 \beta_3 & \beta_2 \alpha_3 & \beta_2 \beta_3 \\
 \end{pmatrix}^T
\]
