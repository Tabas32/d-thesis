% !TEX root = ../thesis.tex

\chapter{Teoretické základy kvantových systémov}

\section{Zakladne definicie}

\subsection{Hilbertov priestor ... atd}
\label{hil_space}
Hilbertov priestor (angl. Hilbert space) je úplný vektorový priestor s operáciou skalárneho súčinu \(<u|v>\), \(u,v\) sú \(N\)-rozmerné vektory s komplexnými zložkami.
Pre takto definovaný priestor platí, že existuje Cauchyho postupnosť, ktorou je dosiahnuteľný ľubovoľný stav, charakterizovateľný \(N\)-rozmerným vektorom \(\phi \in \field{C}^{N}\), ktorý je normalizovaý.

Unitárne zobrazenie (angl. Unitary map) je rotáciou - teda zmenou ortonormálnej bázy.

Kvantový bit je vektor v dvojrozmernom Hilberovom priesotre \(\field{C}^{2}\).
Vo všeobecnosti môžeme vektor \(u = \binom{\alpha}{\beta}\), \(\alpha, \beta \in \field{C}\) a \(u \in \field{C}^{2}\) vyjadriť superpozíciou, teda lineárnou kombináciou základných stavov \(|0>, |1>\), ktoré zodpovedajé klasickým bitom \(0,1\).
Teda \[u = \binom{\alpha}{\beta} = \alpha \binom{1}{0} + \beta \binom{0}{1} = \alpha|0> + \beta|1>,\]
kde monožina \(\{|0>, |1>\} = \{\binom{1}{0}, \binom{0}{1}\}\) je základná množina.
Bázy \(\{|+>, |->\}\) a \(\{|qL>, |qR>\}\) sú ďalšie významné bázy v Hilbertovom priesotre, ktoré sú dosiahnuteľné zo základnej bázy unitárnymi transformáciami.

Stav \(|\phi>\), \(|\phi> \in \field{C}^{2}\) je superpozíciou stavov základnej bázy \(\{|0>, |1>\}\), teda lineárnou kombináciou týchto vektorov.
Superpozícia je daná vzťahom \(|\phi> = \alpha|0> + \beta|1>\), \(\alpha, \beta \in \field{C}, |\alpha|^{2} + |\beta|^{2} = 1\).

\subsection{Systém s jedným kvantovým bitom}
\[\Ket{\psi} = \binom{\alpha}{\beta} = \binom{\alpha}{0} + \binom{0}{\beta} = \alpha \binom{1}{0} + \beta \binom{0}{1} = \alpha \Ket{0} + \beta \Ket{1}, \]
kde \(\alpha, \beta \in \field{C}\) a \(\Ket{\psi} \in \field{C}^{2}\).
Čiže stav kvantového systému \(\Ket{\psi}\) je superpozíciou stavov \(\Ket{0}\) a \(\Ket{1}\).
\[\Ket{\psi} = \binom{\alpha}{\beta}\]
\[\Bra{\psi} = (\alpha^{\dag}\beta^{\dag})\]
\[\Braket{\psi|\psi} = (\alpha^{\dag}\beta^{\dag})\binom{\alpha}{\beta} = \alpha^{\dag}\alpha + \beta^{\dag}\beta = |\alpha|^{2} + |\beta|^{2} = ||\Ket{\psi}||^{2}\]


\subsection{System s viacerymi kvantovymi bitmy}
\subsection{Princip merania}
