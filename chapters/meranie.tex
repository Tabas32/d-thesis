% !TEX root = ../thesis.tex

\chapter{Meranie kvantových obvodov}

Jediným spôsobom ako zistiť skutočný stav kvnatového obvodu je meraním.
Merať možno všetky bity súčasne ako aj jednotlivé kvantové bity samostatne.

\section{Princíp merania kvantových obvodov}

Kvantový bit môže existovať v nekonečnom množstve stavov. Meranie si môžme 
predstaviť ako prevod stavov kvantových bitov do stavu klasického digitálneho
systému \cite{Nie10}. Pre príklad môžeme reprezentovať kvantový stav 
\(\alpha\ket{0} + \beta\ket{1}\) pomocou nulového a excitovaného stavu atómu.
Skutočný kvantový počítač by tak mohol merať tieto stavy. Pri meraní by 
daný atóm skolaboval do jedného zo stavov \(\ket{0}\) alebo \(\ket{1}\).
Pre kolabovanie samozrejme rovnako platí to, že do jednotlivých stavov by
sa atóm dostal s pravdepodobnosťami \(|\alpha|^2\) respektíve \(|\beta|^2\).


Pri každom fyzikálnom meraní nastáva určitá nepresnosť merania. Takisto 
pri meraní môže dokonca nastať zničenie obvodu. To vyplíva z toho, že pri
skolabovanom kvantovom bite nastáva zmena fizykálnych vlastností daného bitu. 

\section{Fiktívne meranie}

Našim cieľom je navrhnúť pravdepodobnostný model, ktorý by umožnil merať
stavy kvantových obvodov aj bez kolabovania jednotlivých kvantových bitov.

\subsection{Experiment 1}
\subsection{Experiment 2}
\subsection{Experiment 3}
