% !TEX root = ../thesis.tex

\chapter{Záver}
Úlohou tejto práce bolo zostaviť programové riešenie problému merania
stavov kvantových bitov počas behu kvantového obvodu. Poskytli sme výstižný
úvod do problematiky kvantových počítačov z pohľadu matematických 
definýcií. Po prečítaní tejto práce by aj laikovi malo byť jasné ako prebieha
kvantový obvod a na akom princípe fungujú merania bitov.

Ešte pred samotnou implementáciou Haskellovského programu sme sa snažili 
detailne priblížiť na akom matematickom princípe postavíme pravdepodobnostný
model. Analýzu návrhu a proces implementácie tohto pravdepodobnostného modelu
sme rozobrali v jednej z kapitol. Vďaka tomu sme mohli ukázať aj riešenie 
zložitejšieho príkladu kvantového obvodu.

Tak ako sme dokázali pri experimentoch, náš pravdepodobnostný model je 
využiteľný pri tvorbe a analýze kvatnových obvodov. Značne urýchľuje výpočty,
ktoré je nutné vykonávať a odvodzovať pri práci s spomínanými obvodmi.
Poskytuje grafickú podobu zmien stavov, čo len ďalej zjednodušuje pochopenie
práce kvantových počítačov.

Do budúcna by bolo dobré vytvoriť grafické prostredie, ktoré by bolo 
nadstavbou tejto práce. Uľahčilo by to prácu, pretože by nebolo nutné poznať
jazyk Haskell pre realizáciu experimentu. Experimenty, ktoré táto práca 
popisuje sú ale dobrým návodom ako tento program využívať. Taktiež obohatenie 
tohto pravdepodobnostného modelu o sofistikovanejší simulátor kvantového stroja
by priniesol presnejšie výsledky, najmä pri obvodoch s previazanými kvantovými
bitmi. V takom prípade by bolo možné úplne nahradiť kvantový simulátor
IBM Quantum Experience, pri návrhu a prvotnom testovaní kvantových obvodoch,
čo by v konečnom dôsledku malo za vplyv rýchlejší rozvoj tejto technológie.
