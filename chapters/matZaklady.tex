% !TEX root = ../thesis.tex

\chapter{Matematické základy kvantových systémov}

Na pochopenie problematiky kvantových počítačov je nutná znalosť aspoň základnej lineárnej algebry.
V tejto kapitole je opísaný matematický aparát využívaný ako teoretický základ celej práce.

\section{Matice}

Maticou typu \(m \times n\) je nazývaná sústava prvkov zapísaných do schémy s \(m\) riadkami a \(n\) stĺpcami, kde \(n,m \in \field{N}\) \cite{Ste18}.
Teda:
\[
A = \begin{bmatrix}
		a_{11} & a_{12} & \dots & a_{1n} \\
		a_{21} & a_{22} & \dots & a_{2n} \\
		{...}							\\
		a_{m1} & a_{m2} & \dots & a_{mn}
     \end{bmatrix}
\]

\subsection{Násobenie matice skalárom}
Toto násobenie je vykonané násobením každého prvku matice danou skalárnou hodnotou \cite{Ste18}.
Majme maticu \(A\) typu \(2 \times 2\) a skalárnu hodnotu \(k\), potom platí
\[
kA = k \begin{bmatrix}
		 a_{11} & a_{12} \\
		 a_{21} & a_{22}
       \end{bmatrix}
= \begin{bmatrix}
	ka_{11} & ka_{12} \\
	ka_{21} & ka_{22}
  \end{bmatrix}
\]
Operácia násobenia matice skalárnou hodnotou je komutatívna, čiže na poradí operandov nezáleží.
Nech \(B\) je matica a \(\alpha, \beta\) sú skalárne hodnoty, potom
\[(\alpha + \beta)B = \alpha B + \beta B,\] \[(\alpha \beta)B = \alpha(\beta B)\]

\subsection{Násobenie matíc}
Nech je daná matica \(A\) typu \(m \times n\) a matica \(B\) typu \(n \times p\), potom výsledná matica \(C = AB\) je typu \(m \times p\) a pre jej prvky platí
\[c_{ij} = \sum_{k=1}^{n} A_{ik}B_{kj} = A_{i1}B_{ij} + \dots + A_{in}B_{nj},\]
kde \(i = 1, \dots ,m\), a \(j = 1, \dots , p\) \cite{Ste18}.
Pre túto operáciu neplatí komutatívnosť.

\subsection{Transpozícia matice}
Ak \(A\) je matica typu \(m \times n\), potom jej transponovaná matica \(A^{T}\) je typu \(n \times m\) a platí \cite{Ste18} \[(A^{T})_{ij} = A_{ji}\]

\subsection{Tenzorový súčin matíc}
Nech \(A\) je matica typu \(m \times n\) a \(B\) je typu \(r \times s\).
Tenzorový súčin alebo Kroneckerov súčin, označený ako \(A \otimes B\) je definovaný ako \cite{Gra81}
 \[
A \otimes B = \begin{bmatrix}
		a_{11}B & a_{12}B & \dots & a_{1n}B \\
		a_{21}B & a_{22}B & \dots & a_{2n}B \\
		{...}							\\
		a_{m1}B & a_{m2}B & \dots & a_{mn}B
     \end{bmatrix}
\]
Nakoľko je \(a_{ij}B\) submatica typu \(r \times s\), je zjavné, že výsledná matica je typu \(mr \times ns\).

\section{Komplexné čísla}
Množinou komplexných čísel \(\field{C}\) je nazývaná množina \(\field{R}^2\) spolu s operáciami sčítania a násobenia. Ľubovoľný prvok \(z = (a, b) \in \field{C}\) je nazývaný komplexné číslo \cite{Tit06}.
Komplexné čísla možno reprezentovať nie len ako usporiadanú dvojicu, ale aj pomocou:
\begin{enumerate}
\item Algebraickej formy \[z = a + bi\], kde \(a,b \in \field{R}\) a \(i^{2} = -1\),
\item Polárnych súradníc \(\rho\) a \(\varphi\), \\
kde \(\rho,\varphi \in \field{R}\) a \(\rho > 0\).
V geometrickej reprezentácii (Obr. \ref{fig:kn}) je \(\rho\) veľkosť vektora \(\vec{Oz}\), kde \(O\) je počiatok súradnicovej sústavy, a \(\varphi\) je uhol medzi osou \(x\) a daným vektorom.
\end{enumerate}
Je zrejme, že pre vyjadrenie pomocou polárnych súradníc platí \(a = \rho \cos \varphi\) a \(b = \rho \sin \varphi\) \cite{Tit06}.
Potom je možné zapísať \[z = \rho e^{i \varphi}\] ,kde \(z \in \field{C}\), \(\rho , \varphi \in \field{R}\) a \(\rho > 1\).
\(e^{i \varphi}\) je komplexná jednotka, inak povedané jej absolútna hodnota je rová 1.
\[|e^{i \varphi}| = 1\]
A z Eulerovho vzťahu platí
\[e^{i \varphi} = \cos \varphi + i\sin \varphi\]

\begin{figure}
\centering
\begin{tikzpicture}[scale=2]
\coordinate (O) at (0,0);
\coordinate (A) at ($({1/sqrt(2)},0)$);
\coordinate (B) at ($({1/sqrt(2)},{1/sqrt(2)})$);

\draw [-latex] (-2,0) -- (2,0) node[below left]{x};
\draw [-latex] (-2,0) -- (2,0) node[above left]{Re$\,z$};
\draw [-latex] (0,-1.5) -- (0,1.5) node[below right]{y};
\draw [-latex] (0,-1.5) -- (0,1.5) node[below left]{Im$\,z$};
\draw (O)--(A)--(B)--cycle;
\draw (O) circle(1);

\tkzLabelSegment[below](O,A){\(a\)}
\tkzLabelSegment[left](A,B){\(b\)}
\tkzLabelSegment[above](O,B){\(\rho\)}

\tkzMarkAngle[size=.4](A,O,B)
\tkzLabelAngle[pos=.3](A,O,B){\(\varphi\)}

\end{tikzpicture}
\caption{Zobrazenie komplexného čísla z: x - reálna os, y - imagináran os}
\label{fig:kn}
\end{figure}

\subsection{Operácie na množine komplexných čísel}
\textbf{Súčet komplexných čísel}
\begin{itemize}
\item \((a + bi) + (c + di) = (a + c) + (b + d)i\)
\item \(\rho_{1}e^{i \varphi_{1}} + \rho_{2}e^{i \varphi_{2}} = \rho_{1}(\cos \varphi_{1} + i\sin \varphi_{1}) + \rho_{2}(\cos \varphi_{2} + i\sin \varphi_{2}) =  (\rho_{1}\cos \varphi_{1} + \rho_{2}\cos \varphi_{2}) + i(\rho_{1}\sin \varphi_{1} + \rho_{2}\sin \varphi_{2})\)
\end{itemize}

\textbf{Násobenie komplexných čísel}
\begin{itemize}
\item \((a + bi)(c + di) = ac + adi + bci - bd = (ac - bd) + (ad + bd)i\)
\item \(\rho_{1}e^{i \varphi_{1}} . \rho_{2}e^{i \varphi_{2}} = \rho_{1} \rho_{2}e^{i(\varphi_{1} \varphi_{2})}\)
\end{itemize}

Operácie rozdielu a podielu sú ľahko odvoditeľné obnobným spôsobom.

\subsection{Základné charakteristiky komplexných čísel}
Nech \(\alpha\) je komplexné číslo \(\alpha = a + bi, \alpha \in \field{C}\).
Potom hovoríme, že \(a,b\) sú zložky komplexného čísla \(\alpha\), pričom \(a\) je reálna a \(b\) je imaginárna zložka.
Pri reprezentácii pomocou polárnych súradníc \(\alpha = \rho e^{i\varphi}\) je \(\rho\) nazývané amplitúda (veľkosť, norma) komplexného čísla a \(\varphi\) je fáza komplexného čísla. \\

Pre komplexné číslo \(\alpha \in \field{C}\) je číslo \(\alpha^{\dag}\) (\(\overline{\alpha}\) alebo \(\alpha^{*}\)) nazývané združeným komplexným číslom (angl. conjugate of complex number) \cite{Tit06}, pričom ak \(\alpha = a + bi\), potom 
\[\alpha^{\dag} = a - bi,\] 
\[\alpha^{\dag} = \rho e^{-i\varphi}.\]
Z geometrickej reprezentácie komplexného čísla na Obr. \ref{fig:kn} je zrejmé, že \(\rho = \sqrt{a^2 + b^2}\).
Bolo už spomenuté, že \(\rho\) sa nazýva aj norma komplexného čísla.
Normu komplexného čísla \(\alpha\) možno označiť aj ako \(|\alpha|\) a platí
\[|\alpha| = \sqrt{\alpha^{\dag}\alpha}.\]
Dôkaz:
\[|\alpha| = \sqrt{\alpha^{\dag}\alpha} = \sqrt{\rho e^{-i\varphi}.\rho e^{i\varphi}} = \sqrt{\rho^{2}} = \rho\]

\section{Vektory}
\label{vektory}

Vektor rozmeru \(n\) je usporiadaný súbor prvkov.
Vo všeobecnosti je možné vektor \(A\) označiť ako 
\[A = \begin{pmatrix}
		a_{1} \\
		a_{2}\\
		\dots \\
		a_{n}
     \end{pmatrix}\]
No je žiadúce označovať vektory pomocou Diracovho (Bra-ket) zápisu.
Čiže vektory \(u = \binom{\alpha}{\beta}\) a \(v = \binom{\gamma}{\delta}\) je lepšie označiť ako 
\[\Ket{\psi_1} = \binom{\alpha_1}{\beta_1}\]
\[\Ket{\psi_2} = \binom{\alpha_2}{\beta_2}\]
Toto označenie popisuje vektory v Hilbertovom priestore (viac v kapitole \ref{hil_space}), pričom platí nasledovné:

Ak \(\Ket{\psi} = \binom{\alpha}{\beta}\) je ket-vektor, potom 
\[\Bra{\psi} = {\binom{\alpha}{\beta}}^{\dag} = (\alpha^{\dag} \beta^{\dag})\]
je  bra-vektor, kde \( ( \alpha, \beta, \alpha^{\dag}, \beta^{\dag} \in \field{C} ) \) a \(\alpha^{\dag}, \beta^{\dag}\) sú združené komplexné čísla ku \(\alpha\) a \(\beta\).
\(\Bra{\psi}\) je teda združenou transpozíciou (angl. transposed conjugate), a platí
\[\Bra{\psi^{\dag}} = \Ket{\psi}\]
\[\Ket{\psi^{\dag}} = \Bra{\psi}\]


\section{Pojmi a definície}
Vektor je \textbf{normalizovaný}, ak jeho norma (veľkosť) je rovná 1.
\[\left\Vert \binom{\alpha}{\beta} \right\Vert = \sqrt{|\alpha|^2 + |\beta|^2}= 1\]

Vektory \(\psi_1\) a \(\psi_2\) sú navzájom \textbf{ortogonálne}, ak ich skalárny súčin je rovný 0. Ortogonálnosť (angl. orthogonality) je v tomto ponímaní teda možné zameniť s kolmosťou.

Dva vektory sú \textbf{ortonormálne}, ak sú zároveň ortogonálne a normalizované.

Pre príklad nech \(\Ket{0} = \binom{1}{0}\) a \(\Ket{1} = \binom{0}{1}\), \( (\Ket{0}, \Ket{1} \in \field{C}^2) \).
Tieto vektory sú ortonormálne, pretože platí 
\begin{enumerate}
\item \(\Braket{0|1} = \Bra{0} . \Ket{1} = \Ket{0^{\dag}} . \Ket{1} = (1 0) . \binom{0}{1} = 0,\)
\item \(\Vert \Ket{0} \Vert^2 = \Braket{0|0} = (1 0) . \binom{1}{0} = 1\) \\
\(\Vert \Ket{1} \Vert^2 = \Braket{1|1} = (0 1) . \binom{0}{1} = 1\).
\end{enumerate}

Pre \textbf{skalárny súčin} dvoch vektorov platí
\[\Braket{\psi_1|\psi_2} = \Bra{\psi_1} . \Ket{\psi_2} = (\alpha_1^{\dag}\beta_1^{\dag}) . \binom{\alpha_2}{\beta_2} = \alpha_1^{\dag}\alpha_2 + \beta_1^{\dag}\beta_2.\]

\textbf{Normu} vektora \(\Ket{\psi}\) pomocou skalárneho súčinu je možné vypočítať ako
\[\Vert \Ket{\psi} \Vert = \sqrt{\Braket{\psi|\psi}},\]
pretože platí \(\Braket{\psi|\psi} = \alpha^{\dag}\alpha + \beta^{\dag}\beta = |\alpha|^2 + |\beta|^2 = \Vert \Ket{\psi} \Vert^2\).

Operácia \textbf{tenzorového súčinu} dvoch vektorov je definovaná ako
\[
\Ket{\psi_1} \otimes \Ket{\psi_2} = \Ket{\psi_1} . \Bra{\psi_2} = \binom{\alpha_1}{\beta_1} . (\alpha_2\beta_2) = 
\begin{pmatrix}
	\alpha_1(\alpha_2\beta_2) \\ 
	\beta_1(\alpha_2\beta_2)
\end{pmatrix} = 
\begin{pmatrix}
	\alpha_1\alpha_2 \quad \alpha_1\beta_2 \\
	\beta_1\alpha_2 \quad \beta_1\beta_2
\end{pmatrix}
\]
