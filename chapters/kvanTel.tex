% !TEX root = ../thesis.tex

\chapter{Kvantová teleportácia}

Predstavme si situáciu, že chceme na diaľku komunikovať. Chceme poslať
informáciu o stave kvantového bitu. Zaznamenať komplexné číslo úplne presne 
na číslocovom počítači nejde a teda nemožno poslať niekomu informáciu o 
presnom stave. Takisto platí, že kvantové bity je nemožné kopírovať či
klonovať. Spôsobom akým je možné riešiť tento problém je takzvaná kvantová
teleportácia.

Medzi účastníkov komunikácie sa rozdelí dvojica previazaných bitov. Ak tieto
kvantové bity nemeriame, zachovajú si svoj stav aj previazanie nehľadiac
na fyzickú vzdialenosť medzi nimi. Teda je možné komunikvať aj na diaľku.

Uveďme si všeobecne známy príklad na kvantovú teleportáciu. Povedzme, že 
Bob chce poslať kvantový bit Alici. Najprv je nutné aby obaja vlastnili
pár previazaných kvantových bitov. Ak teraz chce Bob poslať bit Alici, jediné
čo musí spraviť je aplikovať CNOT hradlo na svoj previazaný bit, ktorý bude 
kontrolovaný kvantovým bitom, ktorý chce odoslať. Potom aplikuje Hadamardovo
hradlo na odosielaný bit a pomeria obe bity. Alici odošle informáciu o
stavoch, ktoré kolabovaním dostal. Alica z tejto informácie vie ako má použiť
hradlá X a Z, tak aby dosiahla rovnaký stav bitu, ktorý povodne vlastnil Bob.

Týmto spôsobom je možné informáciu poslať nakoľko sa jedná o celé čísla.
Takisto nie je porušená veta o kopírovaní ani klonovaní kvantových bitov, 
nakoľko Bob svoj bit stratil.

Ukážme si to na kvantovom obvode. Výtvoríme previazaný pár kvantových bitov
ako na obrázku \ref{tel_c1}. Povedzme, že bit \(q_1\) patrí Bobovi a \(q_2\)
je Alicin.

\begin{figure} 
	\centering 
	\includegraphics[width=.5\textwidth]{figures/tel_c1.png} 
	\caption{Previazanie kvantových bitov na kvantovú teleportáciu.}
    \label{tel_c1}
\end{figure}

Ako bolo spomenuté, na to aby mohol Bob poslať bit \(q_0\), musí aplikovať
CNOT a následne Hadamardovo hradlo tak ako na obrázku \ref{tel_c2}.

\begin{figure} 
	\centering 
	\includegraphics[width=.5\textwidth]{figures/tel_c2.png} 
	\caption{Obvod popisujúci kvantovú teleportáciu.}
    \label{tel_c2}
\end{figure}


