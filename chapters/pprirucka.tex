% !TEX root = ../thesis.tex

\chapter{Používateľská príručka}

\section{Funkcia programu}

Program slúži na meranie kvantového obvodu. Do programu je nutné vložiť 
definíciu kvanotvého obvodu aj s označenými časmi merania. Program vygeneruje
latexové súbory pripravené na preklad pomocou laxeového prekladača. 
Vygenerované sú dva súbory
\begin{enumerate}
    \item Strom zmien stavov kvantových bitov (zn. 1),
    \item Tabuľka výsledkov fiktívnych meraní v daných časových úsekoch (zn. 2).
\end{enumerate}

\section{Systémové požiadavky}

Pre spustenie programu je nutné mať nainštalované: 
\begin{itemize}
    \item interpreter jazyka Haskell,
    \item latex.
\end{itemize}

\section{Inštalácia programu}

Samotný program nie je nutné nijako inštalovať. Programové skripty sú v 
adresáry program, pripravené na spustenie. Je ale nutné si pripraviť prostredie.
Pre inštaláciu jazyka Haskell postupujte podľa návodu na oficiálnej 
stránke \url{www.haskell.org}. Pre operačný systém linux použite svojho
správcu balíkov a nainštalujte haskell-platform. Príklad pre Ubuntu:

\begin{code}
> sudo apt-get install haskell-platform
\end{code}

Na preklad výstupu do formátu .pdf je nutné použiť latex. Pre linux je
inštalácia jednoduchá, pomocou správcu balíkov nainštalujte texlive. Príklad
pre Ubuntu:

\begin{code}
> sudo apt-get install texlive
\end{code}

Pre inštaláciu na Windows postupujte podľa pokynov na \url{http://www.tug.org/texlive/}. Alebo využite online nástroj Overleaf na \url{https://www.overleaf.com/}.

\section{Spustenie programu}

Pre definovanie kvantového obvodu je nutné upraviť súbor QWriter.hs. Vo 
funkcii main sú zakomentované štyry definície obvodov, ktoré boli použité 
v experimentoch diplomovej práce. Odkomentujte jeden alebo definujte nový
obvod rovnakým spôsobom. 

\begin{enumerate}
    \item Nadefinujte vertikálne levely (odporúčanie: pomenujte ich l1..ln),
    \item Uložte levely do listu ako premennú c,
    \item Definujte strom stavov ako premennú st
    \item Definujte tabuľku výsledkov ako rt.
\end{enumerate}

Definícia levelu obsahuje konštruktor Level, list hradiel v danom vertikálnom
levely (E - emplty, X, Y, Z, H, S, Sd, T, Td, Ct - cieľový bit CNOT hradla, 
Cc - kontrólny bit CNOT hradla) a indikátor, či má byť vykonané fiktívne
meranie \underline{za} daným levelom. Príklad:

\begin{code}
Level [E, H, X] True
\end{code}

Definícia stromu stavov obsahuje konštruktor StateTree, počiatočnú 
pravdepodobnosť (odporúčanie: nastaviť na 1), list počiatočných stavov
kvantových bitov (q1, q0, qP, qM, qR, qL alebo ľubovoľný stav definovaný
ako [[Complex Double]], príklad [[0.0 :+ 0.0],[1.0 :+ 0.0]]). 
Veľkosť tohto listu určuje počet kvantových bitov v obvode. Posledný prvok
definície stromu stavov je list podstromov (odporúčanie: nastaviť prázdny list).
Príklad:

\begin{code}
StateTree 1 [q0, q1] []
\end{code}

Definícia tabuľky výsledkov obsahuje konštruktor RT, strom stavov a list 
reprezentujúci samotnú tabuľku (odporúčanie: nastaviť prázdny list). Príklad:

\begin{code}
RT st []
\end{code}

Po definovaní kvantového obvodu je možné spustiť meranie. Spustite interpreter
jazyka haskell ghci v adresáry program.
\begin{code}
> ghci
\end{code}

Načítajte modul QWriter.
\begin{code}
> :l QWriter
\end{code}

Nakoniec spustite funkciu main.
\begin{code}
> main
\end{code}

Program si vypýta názov súboru pre uloženie výsledkov. Odporúčame mať vytvorený
samostatný adresár pre výsledky ale nie je to nevyhnutné. Povedzme, že chceme
uložiť do adresára out, ktorý je v adresáry program. Zadáme názov súboru aj 
s cestou k nemu. Prípona sa doplní samostatne.
\begin{code}
> out/vysledok
\end{code}

Program vytvorí dva súbory pomenované vysledok1.tex a vysledok2.tex. Súbor
s označením 1 obsahuje vygenerovaný strom stavov pomocou latexového balíčka
tikz. Súbor s označením 2 obsahuje tabuľku výsledkov meraní.

Tieto súbory je možné preložiť pomocou latexu. Či už využitím nástroja Overleaf
alebo napríklad pre linux:
\begin{code}
> latexmk -pdf vysledok1
> latexmk -pdf vysledok2
\end{code}
