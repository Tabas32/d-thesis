% !TEX root = ../thesis.tex

\chapter{Ciele prace (Formulácia úlohy)}

Prvou z častí, ktoré je nutné splniť je analýza princípov merania pri 
vykonávaní kvantových programov. Je nutné poskytnúť teoretické informácie
o spôsobe fungovania kvantových počítačov a vysvetliť matematické úkony,
doplnené o praktické príklady, nutné pri meraní zmeny stavov kvantových 
bitov.

Hlavným grom práce bude tvoriť návrh a implementácia zjednodušeného 
kvantového systému. Tento program bude schopný merať stav kvantového systému 
bez kolabovania bitov. Funkcionalita bude postavená na princípoch získaných
z analýzy.

Podstatnou funkcionalitou výsledného programu bude grafické zobrazenie 
vypočítaných pravdepodobností stavov kvantových bitov a prehľad ako sa dané
bity menia počas behu programu.

%Text záverečnej práce musí obsahovať\/ kapitolu s~formuláciou úlohy resp. úloh riešených v~rámci záverečnej práce. V~tejto časti autor rozvedie spôsob, akým budú riešené úlohy a~tézy formulované v~zadaní práce. Taktiež uvedie prehľad podmienok riešenia.
